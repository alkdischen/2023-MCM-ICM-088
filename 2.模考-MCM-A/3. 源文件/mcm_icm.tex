%-------这是COMAP公司发布的2023年美赛官方LaTeX模板
%-------官方仅提供了summary页面的样式,在此基础上,我们构建一个完整的Template!
%-------下载网站:https://www.comap.com/contests/mcm-icm
%-------E-mail:hnuzyx@outlook.com
%%%%%%%%%%%%%%%%%%%%%%%%%%%%%%%%%%%%%%%%
%% MCM/ICM LaTeX Template %%
%% 2023 MCM/ICM           %%
%%%%%%%%%%%%%%%%%%%%%%%%%%%%%%%%%%%%%%%%
\documentclass[12pt]{article}
\usepackage{xeCJK}
\usepackage{verbatim}
\usepackage{geometry}
\geometry{left=1in,right=0.75in,top=1in,bottom=1in}

%%%%%%%%%%%%%%%%%%%%%%%%%%%%%%%%%%%%%%%%
% Replace ABCDEF in the next line with your chosen problem
% and replace 1111111 with your Team Control Number
\newcommand{\Problem}{A}
\newcommand{\Team}{088}
%%%%%%%%%%%%%%%%%%%%%%%%%%%%%%%%%%%%%%%%


\usepackage{hyperref}

\usepackage{newtxtext}
\usepackage{hyperref}
\usepackage{amsmath,amssymb,amsthm}
\usepackage{lipsum}
\usepackage{etoolbox}
\usepackage{booktabs}
\usepackage{float}
\usepackage{indentfirst}%Latex默认在\chapter和\section等章节标题首段不缩进,加入这个宏包可以解决这个问题
\usepackage[pdftex]{graphicx}
\usepackage{xcolor}
\usepackage{fancyhdr}
\usepackage{url}
\lhead{Team \Team}
\rhead{}
\cfoot{}

\newtheorem{theorem}{Theorem}
\newtheorem{corollary}[theorem]{Corollary}
\newtheorem{lemma}[theorem]{Lemma}
\newtheorem{definition}{Definition}

%%%%%%%%%%%%%%%%%%%%%%%%%%%%%%%%
\begin{document}
\graphicspath{{.}}  % Place your graphic files in the same directory as your main document
\DeclareGraphicsExtensions{.pdf, .jpg, .tif, .png}
\thispagestyle{empty}
\vspace*{-16ex}
\centerline{\begin{tabular}{*3{c}}
	\parbox[t]{0.3\linewidth}{\begin{center}\textbf{Problem Chosen}\\ \Large \textcolor{red}{\Problem}\end{center}}
	& \parbox[t]{0.3\linewidth}{\begin{center}\textbf{2023\\ MCM/ICM\\ Summary Sheet}\end{center}}
	& \parbox[t]{0.3\linewidth}{\begin{center}\textbf{Team Control Number}\\ \Large \textcolor{red}{\Team}\end{center}}	\\
	\hline
\end{tabular}}
%%%%%%%%%%% Begin Summary %%%%%%%%%%%
% Enter your summary here replacing the (red) text
% Replace the text from here ...
\begin{center}
%标题
\begin{center}
	\Huge {Art of Balance: Deal for Save and Harvest}
	\vspace{0.4cm}
	
	\normalsize\textbf{Summary}
\end{center}
\vspace{0.2cm}

%下述段落为官方注意事项,写作时注意删除!

\end{center}
% to here


%生成随机段落,写作时删除这排语句即可!
% 当今时代,植物工厂逐渐替代温室,走向了工业化制造的过程中。植物工厂可以为植物生长提供更加可靠和安全的养分供给,从而为产出的植物提供更多丰富的养分,得到更加富有营养的产品


% 但是,当今植物工厂的养分供给较为复杂,对于不同的植物品类,需要不同的养分供给,同时植物工厂提供的养料也较为繁琐,不但包括基本的光照水分,同时也要考虑二氧化碳,空气成分,加热次数,光照时间等等因素。因此我们在本篇论文中,以生菜的生长为例,提供了基本的模型供分析参考。


% 在模型1中,我们得到了基本的单个生菜生长鲜重模型。我们提供了考虑了最基本的因素:光照强度与温度。通过综合考虑光照强度与温度,并将其他参数,例如二氧化碳浓度,含水量等等作归一化处理,我们得到了基于光照强度与温度的生菜生长鲜重模型,作为后续分析处理的基础。


% 在模型2中,我们通过分析种植密度和收获策略,得到了基于20英尺集装箱空间的生菜年收获鲜量模型。该模型着重考虑了种植密度与收获策略两大因素的基本影响。


% 在模型3中,我们加入了工厂的能耗温度,保温材料,朝向,种植密度和光照等因素,来进行工厂耗能的计算,针对人工照明强度和种植密度。在计算能耗时,我们忽略了通风,根据空调的性能指数(COP)讲一个工厂的负荷转换为用电量。


% 在之后的细节分析中,我们综合以上模型,得到了最终可以预测生菜重量和植物工厂能耗的模型。该模型通过综合分析种植密度,朝向,光照,照明强度等多种因素。我们根据该模型分析,得到了适宜生长的模型
In today's era, \textbf{plant factories} are gradually replacing greenhouses in the process of industrial manufacturing. Plant factories can provide a more reliable and safe nutrient supply for plant growth, thus providing more nutrient-rich plants and more nutritious products for the output.


However, the supply of nutrients in today's plant factories is complex, requiring different nutrient supplies for different plant species, and the nutrients provided by plant factories are also complex, including not only basic light and water, but also carbon dioxide, air composition, heating times, light duration, and other factors. Therefore, in this paper, we provide the basic model for the analysis of lettuce growth as an example.


\textbf{In model 1}, we obtained the basic fresh weight model for the growth of individual lettuce. We provided the most basic factors considered: light intensity and temperature. By combining light intensity and temperature, and \textbf{normalizing} other parameters \textbf{such as CO2 concentration, water content, etc.}, we obtained the light intensity and temperature-based lettuce growth weight model as the basis for subsequent analytical treatments.


\textbf{In Model 2}, we obtained a model of the annual fresh weight of lettuce harvest based on a 20-foot container space by analyzing planting density and harvesting strategy. The model focused on the basic effects of two major factors, \textbf{planting density and harvesting strategy.}


\textbf{In Model 3}, we added the plant energy consumption temperature, insulation material, orientation, planting density, and light to \textbf{perform the calculation of plant energy consumption}, for artificial lighting intensity and planting density. In calculating energy consumption, we ignore ventilation and convert a plant load into electricity consumption based on the performance index (COP) of air conditioning.


In the subsequent detailed analysis, we combined the above models to obtain a final model about \textbf{plant factory} that can predict lettuce weight and plant energy consumption. The model was developed by combining various factors such as \textbf{planting density, orientation, light, and lighting intensity.} Based on this model analysis, we obtained \textbf{the model of suitable growth}.

Numerous sources were consulted in the process of refining this article. \textbf{Task 4} required us to analyze the resource management flow and overall energy distribution of the plant. Our model is highly scalable and can be adapted very well to a wide range of influences. These include, but are not limited to, light hours, ventilation resources, etc. All of these factors can be embedded in our model to perfect the analysis of the annual fresh weight of the lettuce plots within a certain space and \textbf{Energy Estimation}. 


We can add all these influencing factors to the calculation of the model and consider that some relevant factors are specified, injecting harvesting strategies, planting density, etc. These strategies are specified from numerous references. These are mentioned at the end of the article. For the calculation of annual fresh weight, our model calculation with \textbf{Dynamic Programming}, has a high similarity with the supporting data of the relevant papers. This is a testament to the high accuracy of the model.Also, for \textbf{Task 5} we chose energy efficient fans to complete the plant production and environment in the plant. All this proves that our model is highly scalable of \textbf{normalization}. We demonstrated the energy-saving potential of the plant by using fans with \textbf{up to 60\% efficiency}.


%关键词
\vspace{0.4cm}
\noindent \textbf{Keywords:}\quad Plant Factory \quad Energy Estimation \quad Dynamic Programming \quad Normalization
%%%%%%%%%%% End Summary %%%%%%%%%%%

%%%%%%%%%%%%%%%%%%%%%%%%%%%%%%
\clearpage
\pagestyle{fancy}
% Uncomment the next line to generate a Table of Contents
%\tableofcontents 
\newpage
\setcounter{page}{1}
\rhead{Page \thepage~of~16}
%%%%%%%%%%%%%%%%%%%%%%%%%%%%%%
%目录
\tableofcontents



\newpage

%正文部分
\section{Introduction}
\subsection{Problem Background}


Lettuce is a long-established herb that is native to the Mediterranean coast of Europe and was first eaten by the ancient Greeks and Romans. It has become popular and spread around the world because of its crispness and sweet taste. In contemporary society, lettuce is also a very popular vegetable, and due to the high demand, scientists are researching how to obtain high lettuce yields with low energy consumption.  


In recent years, the container plant has gained popularity as a new form of plant factory with its low cost, high mobility, and modular capability. Converted from standard containers and equipped with lighting systems, temperature regulation systems, and nutrient supply systems, container plants are a type of plant plant plant that can guarantee high crop yields. The combination of fresh lettuce and the container plant growing method seems to be a great idea for the producer. 
\begin{figure}[h]%插入图片并且加上图片的标题,这是一个模板
    \centering
    \includegraphics[scale=0.18]{{figure/生菜.jpg}}%插入图片的指令
    \includegraphics[scale=0.18]{figure/集装箱.png}
    \caption{Lettuce(Left) and Interior of a Container Plant Factory(Right)}%标题
    \label{Label}
\end{figure}

However, maintaining a suitable growing environment for lettuce requires a lot of energy, which can result in high energy costs if the container plant is not optimized; on the other hand, if the suitable environment is abandoned for energy saving, it will have an impact on the yield and quality of lettuce, and even on sales and thus on income. Therefore, it is necessary to analyze the growth pattern of lettuce and to build a suitable container system accordingly in order to balance the two and reduce the energy consumption per fresh lettuce weight while increasing the crop yield.

\subsection{Restatement of the Problem}



Considering the background information and restricted conditions identified in the problem statement, we need to establish a model that is universal in its applicability to different athletes
and complete the following tasks using the model:    
\begin{itemize}
    \item \textbf{Task 1} asked us to provide a growth model for lettuce that describes the fresh weight of a single lettuce plant as a function of light intensity and temperature. 
    \item In \textbf{Task 2}, we need to give the annual fresh weight of lettuce harvest that can describe a 20-foot container based on the assumed conditions considering planting density and harvesting strategy.
    \item In \textbf{Task 3}, we need to develop a model for calculating the annual air conditioning consumption of a container factory. Besides, We also need a model for calculating the lighting energy consumption of a container factories plant as a 
    function of the intensity of the artificial lighting and planting density.
    \item \textbf{Task 4} required us to develop a model to predict energy consumption per fresh weight of lettuce based on the above model. Also, this task requires us to provide an optimal involved and operational strategy to balance yield and energy consumption.
    \item \textbf{Task 5}  considered temperature regulation through mechanical ventilation. We were required to develop a model to determine the operating strategy of mechanical ventilation and to evaluate its energy-saving potential.
    \item \textbf{Finally}, we need to prepare a one-page article of our findings to MCM Corporation outlining our research.

\end{itemize}
% 任务一要求我们提供一个生菜的生长模型,用于描述单株生菜的鲜重作为光照强度和温度的函数。
% 在集装箱中的空间是有限的,植物也有自己的生长周期。因此在任务二中,我们需要在假设的条件下考虑种植密度和收获策略的基础上给出能够描述20英尺集装箱的生菜收获年鲜重。

% 任务四要求我们根据上述模型,建立一个预测生菜每鲜重耗能的模型,以描述单位产量的能耗如何受到集装箱系统涉及的因数的影响(如种植密度、照明强度、温度等等)。同时该任务还要求我们提供一个最佳的涉及和操作策略,以平衡产量和能源消耗。
% 任务五考虑通过机械通风来调节温度。由于机械通风能耗远低于空调能耗,需要我们开发一个模型确定机械通风的运行策略,并且评估其节能潜力。

\subsection{Our Work}



\begin{figure}[h]
    \centering
    \includegraphics[scale=0.43]{{figure/1.3 OurWork.png}}%插入图片的指令
    \caption{Our Work Structure Graph}%标题
    \label{Label}
\end{figure}

\begin{comment}
我们对任务的背景进行了分析,并建立了模型进行生长预测。我们首先建立了基于单个生菜的生长干重模型。我们对光照和温度两大主要因素进行估计分析,建立了单个生菜的生长模型。我们基于这个单体生长模型,延伸出了有限空间内的年生长鲜重模型。我们思考了种植密度和收获策略,得到了可以在有限集装箱中收获最多生菜鲜重的模型。随后我们加入了集装箱生产的耗能分析,建立了植物工厂完整的生菜生产模型。在这个模型中,我们考虑了照明强度,温度,照明时间等等,并按照生产热量作为分析依据,模拟出了植物工厂的能源消耗。并以机械通风为样本,分析了机械通风的运行策略,并且评估其节能潜力。
\end{comment}

We analyzed the background of the task and built a model for growth prediction. We first developed a growth dry weight model based on individual lettuce. We estimated and analyzed the two main factors, light and temperature, to build a growth model for a single lettuce. We extended the annual growth fresh weight model in a limited space based on this single growth model. We thought about the planting density and harvesting strategy to obtain the model that can harvest the maximum fresh weight of lettuce in a finite container. We then added the energy consumption analysis of container production to build a complete lettuce production model for plant factories. In this model, we considered lighting intensity, temperature, lighting time, etc., and simulated the energy consumption of the plant factory according to the heat production as the basis of analysis. We also analyzed the operation strategy of mechanical ventilation as a sample and evaluated its energy saving potential.


\section{Assumptions and Notations}
\begin{table}[h]
	\begin{center}
		\begin{tabular}{cc}
			\toprule[1.5pt]
			Symbol&Definition\\
			\midrule[1pt]
			\(x_i\)&Evaluation indicator\\
			\({\tilde x_i}\)&Standardized indicators\\
			\({\mu _i}\)&Average value\\
			\(s_i\)&Standard deviation\\
			\({R}\)&Correlation coefficient matrix\\
			\(y_i\)&Principal Components\\
			\(b_i\)&The information contribution of eigenvalue\\
			\(T\)&Composite score\\
			\bottomrule[1.5pt]
		\end{tabular}
	\end{center}
\end{table}
\newpage


\section{Model}

%%%%%%%%%%%%%%%%%%%%%%%%%%%%%%%%%%%%%%%%%%%%%%%%%%%%%%%%

\begin{comment}
task1:

基本参数设定:

假设温度为T(°C),时间t(天),光强l(cd),单个生菜最终鲜重m(g)

基本模型建立:

可以大致认为,光照强度和温度分别影响莴苣的生长,且二者不相互影响.设

\ref{e1}
\begin{equation}\label{e1}
	m=f(T)g(l)
\end{equation}

由常识可知:当T不变时,m在光饱和点之前随l增加而增加,当l达到光饱和点之后,m取值为定值;
当l不变时,m关于T先增后减(酶的活性与T有关).

故可假设$$g(l)=$$

$$f(T)=$$

模型求解:

根据所给数据拟合估计,求出$$g(l)=$$

$$f(T)=$$

由(1)

$$m=f(T)g(l)$$

task2:

基本参数设定:

设土地资源总量为S,单个个体吸收资源s_0,种植密度\rho(棵/m^2),c为工厂面积.

基本模型建立:

根据题意,种植密度只影响土地资源总量S的分配,也即m与个体吸收的土地资源总量呈正比.
在密度较小时,可以认为每个个体充分吸收所有土地资源;密度较大时,假设个体吸收相同的资源.

假设m与s_0呈线性关系$$m=h(s_0)=ks_0+b$$由h(0)=0得b=0.

显然
\ref{e1}
\begin{equation}\label{e1}
	S=cs_0\rho
\end{equation}

模型求解:

根据信息可知,s_0=时,m=500,故k=;

由此结合(2)得到m关于\rho的函数
\ref{e1}
\begin{equation}\label{e1}
	m=\frac{kS}{c\rho}
\end{equation}

task3:

基本参数设定:

设照明时间为t(h).工厂所需温度为$T_0$,外界温度为T,材料密度$\rho(n)$(kg·$m^{-3}$),热导率$\lambda(n)$(W·$m^{-2}$·$K^{-1}$),比热容c(n)(kJ·$kg^{-1}$·$K^{-1}$),总质量$m(n)=V\rho(n)$(V=)(n=1,2,3,4).工厂为长方体,令其水平长方形对角线与南北方向的纬线夹角为$\theta$作为朝向(两条对角线取小的那条).照明灯提供1cd光强所需功率为$p_0$(W·$h^{-1}$).总所需能量为E.注意由于题目考虑因素为温差,所以T的单位取°C或K不影响结果.

基本模型建立:

首先做一个基本假设,假设工厂先因其他条件到达一个温度,而后在不考虑其他影响下空调单独工作使温度到达$T_0$,所以空调消耗可视作以一天为一个单位.

在本题中,t=16.

空调负荷和温差成正比,可以认为
\begin{equation}\label{e1}
	Q=\left\{
\begin{aligned}
k(T-T_0) & , & T>T_0, \\
k(T_0-T) & , & T<T_0.
\end{aligned}
\right.
\end{equation}

$$W_1=\left\{
\begin{aligned}
2.5k(T-T_0) & , & T>T_0, \\
3.5k(T_0-T) & , & T<T_0.
\end{aligned}
\right.$$

灯照明功率为$W_2$,其中提供光强功率为$\frac{W_2}{2}$,散发热量$\frac{W_2}{2}$,

显然$$W_2=lp_0$$
\begin{equation}\label{e1}
	E=365W_1+365tW_2
\end{equation}

工厂体积V=,每立方米空气升高1°C所需热量$q_0=1.1921J$,那么在外界温度为$T_1$的情形时,在工厂内其他因素影响下工厂内温度
$$T=T_1+\frac{Q_{out}}{Vq_0}$$
$$Q_{out}=Q_{sun}+Q_{light}$$

太阳照射的有效面积为$S_{e}=h \cdot x \cdot cos\theta$,根据数据\ h=,x=.

太阳每天照射地球单位面积热量为$q_1=$,工厂表面积$S_{o}=$,墙面厚度\ y=,由此估算工厂外表面温度为
$$T_2=T_1+\frac{S_{e}q_1}{c(n)m(n)}$$
$$Q_{sun}={\lambda}(n)\frac{(T_2-T)S_{o}}{y}$$
$$Q_{light}=t\frac{W_2}{2}$$

task4:

基本参数设定:

设照明时间为t(h),则得到新的m的函数$$m=f(T)g(l,t)$$

基本模型建立:

为平衡产量和能源消耗,认为二者权重一样,在task3中使t=24求出$Q_{max}$,取$m_{max}=500$

定义目标函数
\begin{equation}\label{e1}
	\omega=\frac{m}{m_{max}}-\frac{Q}{Q_{max}}
\end{equation}
求出$\omega_{max}$=,此时各项变量为

\end{comment}

%%%%%%%%%%%%%%%%%%%%%%%%%%%%%%%%%%%%%%%%%%%%%%%%%%%%%%%%%%%%%%%%%%%
\subsection{Model I}

\begin{comment}
为了满足任务一的要求,我们建立了模型一,模型具体的内容已在第二章详细说明。这里阐述一下模型的设置思路与方法。
\end{comment}

In order to meet the requirements of Task 1, we set up Model 1, the details of which have been described in detail in Chapter 2. The ideas and methods of model setup are described here.
\\

\textbf{Basic Parameter Setting for Model I:}

Assuming the temperature is T (°C), time t (days), light intensity l (cd), individual lettuce final fresh weight m (g)
\\

\textbf{Basic Model Building for Model I:}


It can be roughly concluded that light intensity and temperature affect the growth of lettuce respectively, and the two do not affect each other. Establish

%\ref{e1}
\begin{equation}\label{e1}
	m=f(T)g(l)
\end{equation}

It can be known from common sense: when T is unchanged, m increases with the increase of l before the light saturation point, and when l reaches the light saturation point, the value of m is fixed;
When l does not change, \textit{m} increases and then decreases with respect to T (enzyme activity is related to T).\\
Our reference data is shown below.
\begin{figure}[h]
    \centering
    \includegraphics[scale=0.5]{{figure/Model I Data.png}}%插入图片的指令
    \caption{The Growth Trend of a Lettuce in 60 Days in Different Light Intensity and Temperature}%标题
    \label{Label}
\end{figure}


Looking at our data graphs, we can see that the temperature formulas are in a quadratic polynomial relationship, while the light intensity formulas are in an exponential relationship therefore we make the following assumptions.
\begin{equation}\label{e1}
	m=f(T)g(l)
\end{equation}

\begin{equation}\label{e1}
	g(l)=  a_1+b_1e^{c_1(l+d_1)}
\end{equation}
\begin{equation}\label{e1}
	f(T)= a_2\mathop{{T}}\nolimits^{{2}}+b_2T+c_2 
\end{equation}

Where \textit{a,b,c,d }are all parameters.
\\
\\
\newpage


\textbf{Model Solving for Task 1:}

Based on the given data, the estimation is fitted
\begin{equation}\label{e1}
	g(l)= 10.702-6.937e^{-\frac{l-0.22}{4.02}}
\end{equation}

\begin{equation}\label{e1}
	f(T)= -0.0009\mathop{{T}}\nolimits^{{2}} + 0.3171T + 11.091
\end{equation}
 \label{el}

from(1)


\begin{equation}\label{e1}
	m=f(T)g(l)= (10.702-6.937e^{-\frac{l-0.22}{4.02}})(-0.0009\mathop{{T}}\nolimits^{{2}} + 0.3171T + 11.091)
\end{equation}

We performed data validation with the model and the results are as follows.

\begin{figure}[h]
    \centering
    \includegraphics[scale=0.7]{{figure/Model I Vadiation.png}}%插入图片的指令
    \caption{Model I Validation}%标题
    \label{Label}
\end{figure}

We found that most of the vegetables were concentrated between 150g and 200g, which is in good agreement with the facts. Also the model can be helpful for Task 2.
%%%%%%%%%%%%%%%%%%%%%%%%%%%%%%%%%%%%%%%%%%%%%%%%%
\newpage
\subsection{Model II}
\begin{comment}
对于任务二,我们建立模型二。基本设置如下。
\end{comment}

For Task 2, we build Model 2. The basic setup is as follows\\

\textbf{Basic Parameter Setting for Model II:}

Let the total amount of land resources be S, the $s_0$\ of resources absorbed by a single individual, the planting density $\rho$(tree/$m^2$), and\ c\ be\ the area\ of\ the\ factory.\\

\textbf{Basic Model Building for Model II:}

According to the title, planting density only affects the distribution of total land resources S, that is, m is proportional to the total amount of land resources absorbed by individuals.
At low densities, it can be considered that each individual fully absorbs all land resources; When the density is large, it is assumed that individuals absorb the same resources.

Suppose m has a linear relationship with $s_0$:
\begin{equation}\label{e1}
	m=h(s_0)=ks_0+b 
\end{equation}


From\ h(0)=0 to b=0.

Apparently,
%\ref{e1}
\begin{equation}\label{e1}
	S=cs_0\rho
\end{equation}
\\

\textbf{Model Solving for Task 2:}

According to the information, when $s_0$=103, m=500, c=14.884, so k≈4.85;

This combines (2) to obtain the function of m with respect to $\rho$
%\ref{e1}
\begin{equation}\label{e1}
	m=\frac{kS}{c\rho}=\frac{4.85S}{14.884\rho}≈0.326\frac{S}{\rho}
\end{equation}
\\
We next used the model to make predictions.We assume that the number of lettuce per unit area is 3. Based on the container size of 20 feet, we predict the following results.

\begin{figure}[h]
    \centering
    \includegraphics[scale=0.7]{{figure/Model II Prediction.png}}%插入图片的指令
    \caption{Model II Prediction for Random Light Intensity }%标题
    \label{Label}
\end{figure}

We predicted Model II with random light intensity and temperature, and the prediction results were more consistent with the relevant data we obtained, within 10 percent error, with a high degree of confidence.


\subsection{Model III}

\textbf{Basic Parameter Setting for Model III:}


The required temperature of the plant is $T_0$, the outside temperature is T, the material density is $\rho(n)$(kg·$m^{-3}$), thermal conductivity $\lambda(n)$(W·$m^{-2}$·$K^{-1}$), specific heat capacity c(n)(kJ·$kg^{-1}$·$K^{-1}$), total mass $m(n)=V\rho(n)$($V=s_1\cdot\lambda(n)$, $s_1$ is the surface area of the container, and n=1,2,3,4). Let its horizontal rectangular diagonal and the latitude line in the north-south direction be $\theta$ as the orientation (the smaller of the two diagonals).The power required for the lamp to provide 1cd light intensity is $p_0$(W·$h^{-1}$).The total required energy is E. Note that since the topic consideration factor is the temperature difference, the unit of T takes °C or K does not affect the result.\\

\\
\\ \hspace*{\fill} \\

\textbf{Basic Model Building for Model III:}


First of all, make a basic assumption, suppose that the factory first reaches a temperature due to other conditions, and then the air conditioner works alone to make the temperature reach $T_0$ without considering other influences, so the air conditioning consumption can be regarded as a unit of a day.

The air conditioning load is proportional to the temperature difference, it can be considered
\begin{equation}\label{e1}
	Q=\left\{
\begin{aligned}\label{el}
k(T-T_0) & , & T>T_0, \\
k(T_0-T) & , & T<T_0.
\end{aligned}
\right.
\end{equation}
\begin{equation}\label{e1}
W_1=\left\{
\begin{aligned}
2.5k(T-T_0) & , & T>T_0, \\
3.5k(T_0-T) & , & T<T_0.
\end{aligned}
\right.
\end{equation}


The lamp lighting power is $W_2$, where the light intensity power provided is $\frac{W_2}{2}$, and the heat dissipated $\frac{W_2}{2}$,

Apparently 
\begin{equation}\label{e1}
	W_2=lp_0
\end{equation}

\begin{equation}\label{e1}
	E=365W_1+16 \times 365W_2
\end{equation}


Factory volume $V_0$=39.14$m^3$,the heat required to increase 1 °C per cubic meter of air is $q_0=1.1921J$, so when the outside temperature is $T_1$, the temperature in the factory is affected by other factors in the factory
\begin{equation}\label{e1}
	T=T_1+\frac{Q_{out}}{V_0q_0}
\end{equation}

\begin{equation}\label{e1}
	Q_{out}=Q_{sun}+Q_{light}
\end{equation}


The effective area of solar irradiation is $S_{e}=h \cdot x \cdot cos\theta$, according to the data  h=2.63m,x=6.10m.

The heat per unit area of the sun hitting the earth is $q_1=1367J$ per second, the factory surface area is $S_{o}=58.64m^2$, and the wall thickness  $y=2.2mm$, from which the temperature of the outer surface of the factory is estimated
\begin{equation}\label{e1}
	T_2=T_1+\frac{S_{e}q_1}{c(n)m(n)}
\end{equation}


\begin{equation}\label{e1}
	Q_{sun}={\lambda}(n)\frac{(T_2-T)S_{o}}{y}
\end{equation}

\begin{equation}\label{e1}
	Q_{light}=16 \times \frac{W_2}{2}=8{W_2}
\end{equation}


%%%%%%%%%%%%%%%%%%%%%%%%%%%%%%%%%%%%%%%%%%%%%%

\subsection{Application 1: Fresh weight energy model for lettuce}

Tasks 4 requires us to calculate the overall energy consumption separately. We performed separate analytical estimates using the models described above. The analysis process is as follows.


Basic parameter setting:

If the illumination time is t(h), we get the new function of m
\begin{equation}\label{e1}
	m=f(T)g(l,t)
\end{equation}


Basic model building:

In order to balance the yield and energy consumption, considering that the weights of the two are the same, let t=24 find $Q_{max}$ in task3, and take $m_{max}=500$

Define the objective function
\begin{equation}\label{e1}
	\omega=\frac{m}{m_{max}}-\frac{Q}{Q_{max}}
\end{equation}

Find $\omega_{max}$=307, which satisfies our needs above.

\begin{figure}[h]
    \centering
    \includegraphics[scale=0.55]{{figure/Model III Prediction and Vadiation.png}}%插入图片的指令
    \caption{Model III Prediction and Vadiation }%标题
    \label{Label}
\end{figure}

We used the relevant data to compare the annual fresh weight and the final output for prediction, and the model output always remained within a certain error range. The stability and extensibility of the model were demonstrated.

%%%%%%%%%%%%%%%%%%%%%%%%%%%%%%%%%%%%%%%%%%%%%%%%%%%%%%%%

\subsection{Application 2: Mechanical ventilation operation model}

Task 5 asked us to consider the case of mechanical ventilation. Since natural ventilation is not stable, the container is mechanically ventilated with fan-driven ventilation. We assume 2 air changes per hour, then 17,520 air changes are required in 1 year time. We combine this with Model III and simply add the energy required for mechanical air changes to the resource consumption of the plant.

According to our survey, centrifugal fans with a backward curved outer cover (backward tilting) have the highest efficiency rating, with a power range P between 10 and 50$kW$ and less than 1kw at static, with an efficiency of $\eta$.
\begin{equation}\label{e1}
	\eta =1.1\ln{P}-2.6+N
\end{equation}
where $N$ means the amount of centrifugal fans.

Considering that our plant plant is a standard 20-foot container, we thought that two fans could be installed to meet the overall ventilation needs. Therefore, the final result of the installation was two centrifugal fans with backward curved outer cover (backward tilting), working twice per hour. Therefore the total energy consumed in the end was$W_{fans}$.
\begin{equation}\label{e1}
	W_{fans} =\frac{(P_{static}\times5+P_{dynamic})}{6}\times24\times365\approx7329.2kW
\end{equation}

After our final analysis, we can conclude that mechanical ventilation requires approximately nearly 7,400 kW of power consumption, which is far less than air conditioning blowing. Although higher than natural ventilation, mechanical ventilation can ensure product quality while enhancing air management and more in line with fire codes. It is a more energy-saving and environmentally friendly ventilation method. It has a very high energy-saving potential.

%%%%%%%%%%%%%%%%%%%%%%%%%%%%%%%%%%%%%%%%%%%%%%%%%%%%%%%%




%下面是之前模板里的常用公式,这里给它们注释掉了
\begin{comment}
\subsection{Equations}

%数学公式:
This is an equation:$$\frac{x}{y}=\frac{\sqrt[3]{x}}{\ln a}$$

This is an equation with number:(Heat Equation)\ref{e1}
\begin{equation}\label{e1}
	c\rho\frac{\partial u}{\partial t}=k\nabla^2u+f
\end{equation}

Matrix:
$$
\left( \begin{matrix}
1&		2&		3&		4\\
a&		b&		c&		d\\
x&		y&		z&		w\\
\alpha&		\beta&		\gamma&		\varphi\\
\end{matrix} \right) 
$$

Integral:

$$
\oint_l{pdx+qdy+rdz}
$$

Some Equation:
\begin{align*}
\cosh x = \frac {1}{2} (e^x + e^{-x}) &= \sum_{n = 0}^{\infty} \frac {x^{2n}}{(2n)!} \\
\sinh x = \frac {1}{2} (e^x - e^{-x}) &= \sum_{n = 0}^{\infty} \frac {x^{2n + 1}}{(2n + 1)!} \\
e^x &= \sum_{n = 0}^{\infty} \frac {x^n}{n!} = \lim_{n\to\infty} \left (1+\frac{x}{n} \right )^n\\
\end{align*}

\subsection{Figures}
This is a figure\ref{fig:comap-logo}:
\begin{figure}[h]
	\centering
	\includegraphics[width=0.7\linewidth]{"figure/comap logo"}
	\caption{Comap logo}
	\label{fig:comap-logo}
\end{figure}

\end{comment}

\section{Extend Our Model}


%%%%%%%
%我们任务一中的模型中选取的是比较小范围的温度来进行模型的拟合,因为上海崇明岛是属于亚热带季风性气候,因此全年的温差不大,大部分时候的温度集中在10摄氏度到30摄氏度内。不过如果考虑到可能在冬季的零下低温和夏季的高温对植物中各种酶的损伤,从而导致的后续植物生长的变缓甚至停滞,本模型就没有涉及对这一方面的讨论;不过由于我们后来要研究的是在集装箱内的植物生长模型,那么这一不足就显得无关紧要了。对于光照也是同样的道理,讨论适宜范围的光照强度即可。   
%任务二中,我们考虑了单个个体吸收的土地资源,因为单个个体受到单株土地占有量(即总土地面积除以种植密度)的影响并非线性关系,因此需要先把这种影响单独拿出来进行分析。当然,对于不同的土地或者培养皿其对应的函数是不一样的,我们这里是选了一种特定的培养环境进行分析;如果有更换培养环境的需要则可以进一步分析:如不使用土地栽培而是用培养皿分瓶栽培,那么这里对于种植密度的讨论就只局限于光照竞争而不考虑营养成分竞争了,当然土地培养和培养皿培养本身的成本就有不同,这里我们还是选用成本较低的土地培养。同时在本模型中我们考虑的是使得在收获100-200g生菜的条件下的最大收获量,那么对于蔬菜的品质、口味就没有过多的考虑:当然了,如果在某一个生长阶段的生菜口味较好,容易出售的话,我们也可以据此调整收获时间。

%在任务三中,我们终于引入了温控和光控系统对于集装箱的温度的调节。此时的光强和温度都是根据模型一里的最优值进行设定的;当然,对于植物来说,适宜的温度和光强不只有一组数值,而是一个范围,因此如果对降低能源损耗有着更高的要求,那么我们可以给温度设定一个区间值,比如低温的时候让箱内温度大于等于一个最小值,高温的时候让温度小于等于一个最大值,如果外部气温处于该温度区间则可以减少甚至停止空调的使用。本次的集装箱使用的是不透光的集装箱,因而生菜所需光照均由内部灯光解决:如果考虑使用玻璃外壳集装箱,那么光照的耗能就会相应地减少,但是在高温暴晒天气下的调温功率就会显著上升。考虑到空调的耗电远高于电灯的耗电,那么我们还是尽量以增加光照能耗为代价去尽量减少空调的能耗,因而不使用透明集装箱。

%%%%%%%
%Task 1
\begin{itemize}
    \item \textbf{Model I:}
    
    The model in \textbf{Task 1} was fitted to a relatively small range of temperatures because Chongming Island, Shanghai, has a subtropical monsoonal climate, so there is little temperature difference throughout the year, with most of the time the temperature ranges from 10 to 30 degrees Celsius.

    However, if we consider the possible damage to various enzymes in the plants due to subzero temperatures in winter and high temperatures in summer, which may slow down or even stagnate the subsequent plant growth, this aspect is not discussed in this model; however, since we will later study the plant growth model in containers, this deficiency is irrelevant. The same holds for light, and it is sufficient to discuss the appropriate range of light intensities. 
    
    \item \textbf{Model II:}
    
    In \textbf{Task 2}, we considered the land resources absorbed by individual individuals, because the influence of individual individuals by the amount of land occupied by a single plant (i.e., a total land area divided by planting density) is not linearly related, so this influence needs to be taken out separately for analysis first. Of course, the corresponding function is different for different land or Petri dishes, and we have chosen a specific cultural environment for analysis here; if there is a need to change the cultural environment, then further analysis can be done: if instead of land cultivation, Petri dishes are used for vial cultivation, then the discussion of planting density here is limited to light competition without considering the nutrient competition. 

    Of course, the cost of land culture and Petri dish culture itself is different, so here we still choose the lower cost of land culture. Also in this model, we consider the maximum harvest volume under the condition of harvesting 100-200g of lettuce, so we do not consider too much about the quality and taste of the vegetables: of course, if the lettuce is at a certain growth stage tastes better and is easy to sell, we can adjust the harvest time accordingly.
    
    \item \textbf{Model III:}
    
    %Task 3
    In \textbf{Task 3}, we finally introduced the temperature control and light control system for the regulation of the container temperature. At this time, the light intensity and temperature are set according to the optimal values in Model 1; of course, for plants, the appropriate temperature and light intensity is not just a set of values, but a range, so if there are higher requirements for reducing energy losses, then we can set an interval value for the temperature, such as low temperature so that the temperature inside the box is greater than or equal to a minimum value, a high temperature so that the temperature is less than or equal to A maximum if the external temperature is in the temperature range can reduce or even stop the use of air conditioning.
    
    The container used in this case is opaque, so the light required for lettuce is solved by the internal light: if we consider using a glass shell container, the energy consumption of light will be reduced accordingly, but the power of temperature regulation will be significantly increased in hot and sunny weather. Considering that the power consumption of air conditioning is much higher than the power consumption of electric lights, then we still try to increase the energy consumption of light at the expense of minimizing the energy consumption of air conditioning, and therefore do not use transparent containers.
    
\end{itemize}
  

%Task 2


\section{Discussion}
% 讨论一下模型的延展性
% 优点
% 我们的模型考虑充分,复杂。考虑了多种情况,并按照最终成果进行了诸多分析
% 缺点
% 可以考虑更多
\subsection{Strength}
The advantages of this model are:
\begin{itemize}
    \item Simple and easy to understand, through some functional relations to solve a variety of complex relationships, eliminating complex and redundant and not too helpful to the actual application of the scene of parameters, so that users can use this model without difficulty.
    \item The fitted data of the model parameters are all top papers found on major academic platforms, which are reliable and authoritative so the fitted results are also very realistic.
    \item The relationship between sunlight and container placement angle is fully considered so that the placement angle also becomes a point to be considered rather than a casual treatment, which is evidence of the degree of insight.
    \item We have taken great care to find papers on various fields, and have used time-tested classical formulas and theorems in the paper, thus ensuring the feasibility of the model.
\end{itemize}
\subsection{Weakness}
The model also has some shortcomings:
\begin{itemize}
    \item Only the shape of the container (rectangular) is considered for its influence on the heat transfer with the external environment and the ventilation and air exchange, and the influence of different weather on these two parts is not fully considered. For example, rainy days will bring the rise of air humidity and water droplets on the surface of the container.
    \item It does not take into account the decline in the efficiency of temperature regulation and light provision caused by the gradual aging of these devices in the course of use, thus increasing energy consumption; and does not take into account the maintenance and replacement of equipment in the long term.
\end{itemize}
\subsection{Future Work}
Considering the shortcomings of the model, we also hope that some improvements can be made in the future:
\begin{itemize}
    \item The weather should also be considered: including consideration of the impact of air humidity on mechanical ventilation and temperature control effect during rainy and snowy days, as well as the existence of water droplets and snowflakes attached to the surface of the container in the evaporation and sublimation on the container surface heat transfer; at the same time, we should also consider the impact of possible typhoon weather in Chongming Island, Shanghai, such as the impact of groundwater on heat dissipation, and how to make The design and manufacture of containers can be more adapted to the climate of the region.
    \item The depreciation of the equipment also needs to be considered. A model of the change of equipment efficiency with time can be established, and the cost of maintenance and renewal can be counted, so that the decision of whether to repair or renew the equipment can be made at an appropriate time, thus making the model more realistic.
\end{itemize}

\newpage
\appendix
\section{References}

\begin{thebibliography}{99}
	\bibitem{1}Graamans, L., Baeza, E., Van Den Dobbelsteen, A., Tsafaras, I. and Stanghellini, C., Plant factories versus greenhouses: Comparison of Resource Use Efficiency, Agricultural Systems, 160, pp.31-
43, 2018.
	\bibitem{2}Van Henten, E.J., Validation of a Dynamic Lettuce Growth Model for Greenhouse Climate Control, Agricultural Systems, 45(1), pp.55-72, 1994.
	\bibitem{3}Al-Homoud, M.S., Computer-Aided Building Energy Analysis Techniques. Building and Environment, 36(4), pp.421-433, 2001.
	\bibitem{4}Typical Meteorological Year Data: https://www.ladybug.tools/epwmap/
	\bibitem{5}Shenzhen Institute of Standards and Technology:
	https://tbt.sist.org.cn/mbsc\_106/omsc\_111/cpj
	n\_376/erpzl/201004/t20100426\_172872.html
\end{thebibliography}




%%%%%%%%%%%%%%%%%%%%%%%%%%%%%%
\end{document}
\end