
\section{Discussion}

\subsection{Strength}
\begin{itemize}
    \item \textbf{Our Model} has a high degree of scalability and modularity, and is highly practical for handling special hypothetical scenarios and future predictions. It quantifies abstract concepts and provides a demonstration and guidance for decision-making in problem-solving.
    \item \textbf{Model I} fully utilizes the AHP method for decision analysis, and simultaneously carries out innovative and summarizing comprehensive planning for the 17 sustainable development goals, providing reference solutions for future decision-making regarding sustainable development goals.
    \item \textbf{Model II} integrates the use of K-Means model for comprehensive decision analysis. K-Means can not only perform integrated processing but also preprocess the complex data to achieve a more convenient decision-making model. This is a highly innovative combination and an excellent solution for multi-option decision-making.
    \item \textbf{Model III} has made a secondary innovation based on Model 2, optimizing the decision direction of the decision-making layer, ensuring the predictability and overall fluidity of the model, and possessing high practicality for long-term forecasting of the problem.
\end{itemize}



\subsection{Weakness}
\begin{itemize}
    \item It is preferable to use the Delphi method to obtain the weights of the decision layer in the AHP model. However, due to the tight schedule of the competition, we had to refer to relevant materials for analysis, which may inevitably lead to errors.
    \item The K-Means method is best to undergo multiple training attempts to obtain the final clustering solution. Due to the tight schedule of the competition and the insufficient relevant data, we were unable to perform multiple iterative training for comparison. Instead, we adopted the clustering solution that performed well in the test data.
\end{itemize}


\subsection{Future Work}
Considering the shortcomings of the model, we also hope that some improvements can be made in the future:
\begin{itemize}
    \item \textbf{Data preprocessing:} In K-Means, data preprocessing has an important impact on the accuracy and stability of clustering results. Data cleaning, normalization, and dimensionality reduction can be used to process data and improve the effectiveness of the model.
    
    \item \textbf{Parameter selection:} In K-Means and AHP methods, the selection of model parameters also has a significant impact on the accuracy of the results. For example, in K-Means, the optimal number of clusters can be determined by methods such as the elbow method and silhouette coefficient. In AHP, the subjectivity of decision-makers and the completeness of information collection also affect the results and require careful consideration.
    
    
    \item \textbf{Model integration:} In K-Means and AHP methods, more advanced model integration methods can be used to improve the effectiveness of the model. For example, ensemble learning and deep learning can be used to integrate the results of multiple models and improve prediction accuracy and stability.
    
    \item \textbf{Model evaluation:} Model evaluation is a critical step in K-Means and AHP methods, and various indicators can be used to evaluate the effectiveness of the model to determine if it meets expected performance. For example, in K-Means, indicators such as SSE and silhouette coefficient can be used to evaluate the accuracy of clustering results. In AHP, consistency indicators and consistency ratios can be used to evaluate the reliability of decision-making results.
\end{itemize}
